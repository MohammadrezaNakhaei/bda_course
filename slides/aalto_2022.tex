\documentclass[english]{beamer}
%\documentclass[finnish,english,handout]{beamer}

% Uncomment if want to show notes
%\setbeameroption{show notes}

\mode<presentation>
{
  \usetheme{Warsaw}
  % oder ...

  %\setbeamercovered{transparent}
  % oder auch nicht
}


%\usepackage[pdftex]{graphicx}
\usepackage[T1]{fontenc}
\usepackage[latin1]{inputenc}
%\usepackage[T1,mtbold,lucidacal,mtplusscr,subscriptcorrection]{mathtime}
\usepackage{times} % times
\usepackage{amsmath}
\usepackage[varqu,varl]{inconsolata} % typewriter
\usepackage{microtype}
\usepackage{url}
\urlstyle{same}

\hypersetup{%
  bookmarksopen=true,
  bookmarksnumbered=true,
  pdftitle={Use of reference models in model selection},
  pdfsubject={Bayesian model selection},
  pdfauthor={Aki Vehtari},
  pdfkeywords={},
  pdfstartview={FitH -32768},
  colorlinks=true,
  linkcolor=blue,
  citecolor=blue,
  filecolor=blue,
  urlcolor=blue
}

% \definecolor{hutblue}{rgb}{0,0.2549,0.6784}
% \definecolor{midnightblue}{rgb}{0.0977,0.0977,0.4375}
% \definecolor{navyblue}{rgb}{0,0,0.5}
% \definecolor{hutsilver}{rgb}{0.4863,0.4784,0.4784}
% \definecolor{lightgray}{rgb}{0.95,0.95,0.95}
% \definecolor{section}{rgb}{0,0.2549,0.6784}
% \definecolor{list1}{rgb}{0,0.2549,0.6784}
% \renewcommand{\emph}[1]{\textcolor{navyblue}{#1}}

\graphicspath{../luku1}

\pdfinfo{
          /Title      (Bayesian data analysis)
          /Author     (Aki Vehtari) %
          /Keywords   (Bayesian probability theory, Bayesian inference, Bayesian data analysis)
}


\parindent=0pt
\parskip=8pt
\tolerance=9000
\abovedisplayshortskip=0pt

\setbeamertemplate{navigation symbols}{}
\setbeamertemplate{headline}[default]{}
\setbeamertemplate{headline}[text line]{\insertsection}
\setbeamertemplate{footline}[frame number]


\def\o{{\mathbf o}}
\def\t{{\mathbf \theta}}
\def\w{{\mathbf w}}
\def\x{{\mathbf x}}
\def\y{{\mathbf y}}
\def\z{{\mathbf z}}

\DeclareMathOperator{\E}{E}
\DeclareMathOperator{\Var}{Var}
\DeclareMathOperator{\var}{var}
\DeclareMathOperator{\Sd}{Sd}
\DeclareMathOperator{\sd}{sd}
\DeclareMathOperator{\Bin}{Bin}
\DeclareMathOperator{\Beta}{Beta}
\DeclareMathOperator{\logit}{logit}
\DeclareMathOperator{\N}{N}
\DeclareMathOperator{\U}{U}
\DeclareMathOperator{\BF}{BF}
%\DeclareMathOperator{\Pr}{Pr}
\def\euro{{\footnotesize \EUR\, }}
\DeclareMathOperator{\rep}{\mathrm{rep}}


\title[]{Bayesian data analysis}
\subtitle{Practical matters}

\author{Aki Vehtari}

\institute[Aalto University]{}

\begin{document}

\section{Course contents}


\begin{frame}
  \frametitle{Bayesian data analysis (Aalto fall 2022)}  %
  \framesubtitle{}
  \begin{itemize}
  \item Book: Gelman, Carlin, Stern, Dunson, Vehtari \& Rubin: Bayesian Data
    Analysis, Third Edition. {\footnotesize (online pdf available)}
  \item The course website has more detailed information than these slides\\
    {\small\url{https://avehtari.github.io/BDA_course_Aalto/Aalto2022.html}}
  \item Timetable: see the course website
  \item TAs: Anna Riha, Elena Shaw, Kunal Ghosh, Andrew Johnson, Noa
    Kallioinen, David Kohns, Leevi Lindgren, Teemu Sailynoja, Niko Siccha
    % \item Oodi mentions TA session this week on Wednesday, but that
    %   has cancelled.
    \end{itemize}
    \vspace{-\baselineskip}
 \begin{center}
   \includegraphics[width=2.6cm]{figs/BDA3.jpg}
 \end{center}

\end{frame}

\begin{frame}
  \frametitle{Bayesian data analysis}  %
  \framesubtitle{Pre-requisites}
  \begin{itemize}
  \item Basic terms of probability theory
    \begin{itemize}
    \item probability, probability density, distribution
    \item sum, product rule, and Bayes' rule
    \item expectation, mean, variance, median
    \end{itemize}
  \item Some algebra and calculus
  \item Basic visualisation techniques (R or Python)
    \begin{itemize}
    \item histogram, density plot, scatter plot
    \end{itemize}
  \end{itemize}

  These will be tested with the first assignment round

\end{frame}

\begin{frame}
  \frametitle{Bayesian data analysis}  %
  \framesubtitle{Pre-requisites}
  \begin{itemize}
  \item What to do if the course seems to be too difficult
    \begin{itemize}
    \item refresh your memory on pre-requisites (see the course web
      site for some links)
    \item ask for help
    \item consider reading Regression and Other Stories \url{https://avehtari.github.io/ROS-Examples/}
    \item consider reading Statistical rethinking + watching videos \url{https://xcelab.net/rm/statistical-rethinking/}
    \end{itemize}
  \end{itemize}

\end{frame}


\begin{frame}
  \frametitle{Bayesian data analysis}  %
  \framesubtitle{Different learning styles}

  \begin{itemize}
  \item Reading
  \item Listening lectures
  \item Solving problems
    \begin{itemize}
    \item mathematical derivations
    \item programming
    \end{itemize}
  \end{itemize}
  
\end{frame}

% \begin{frame}
%   \frametitle{Bayesian data analysis}  %
%   \framesubtitle{Course contents}
%   \begin{itemize}
%   \item Background (Ch 1)
%   \item Single-parameter models (Ch 2)
%   \item Multiparameter models (Ch 3)
%   \item Computational methods (Ch 10)
%   \item Markov chain Monte Carlo (Ch 11--12)
%   \item Stan and probabilistic programming
%   \item Hierarchical models (Ch 5)
%   \item Model checking (Ch 6)
%   \item Evaluating and comparing models (Ch 7)
%   \item Decision analysis (Ch 9)
%   \item Large sample properties and Laplace approximation (Ch 4)
%   \item In addition you learn workflow for Bayesian data analysis
%   \end{itemize}
  
% \end{frame}

\begin{frame}
  \frametitle{Bayesian data analysis}  %
  \framesubtitle{Example analyses}
  \begin{itemize}
  \item Treatment/control
    \begin{itemize}
    \item randomize patients to treatment or control
    \item is the treatment effective?
    \end{itemize}
    \pause
  \item Continuous valued treatment
    \begin{itemize}
    \item randomize patients with different dosages
    \item which dosage is sufficient without too many side effects?
    \end{itemize}
    \pause
  \item Different effects for different patients?
    \begin{itemize}
    \item Is the treatment effect different for male/female,
      child/adult, light/heavy, ...
    \end{itemize}
  \end{itemize}

\end{frame}

\begin{frame}
  \frametitle{Bayesian data analysis}  %
  \framesubtitle{Computer exercises}
  \begin{itemize}
  \item Basic visualisation techniques
  \item Binomial distribution -- Algae
  \item Normal distribution -- Windshield
  \item Difference between binomials -- Treatment/control
  \item Difference between normals -- Windshield
  \item Generalized linear model (GLM) + importance sampling -- Bioassay
  \item GLM + Metropolis + convergence diagnostics -- Bioassay
  \item GLM + Bioassay + Stan
  \item Linear model + Stan
  \item Hierarchical model + Stan
  \item Model seletion + Stan
  \end{itemize}

\end{frame}

\begin{frame}{Stan}
  
  Stan is a probabilistic programming framework and ecosystem

  40+ developers, 100+ contributors, 100K+ users

  R, Python, Julia, Scala, Stata, Matlab, command line interfaces
  
  More than 140 R packages using Stan

  Many packages to support diagnostics and workflow
  
  % \includegraphics[width=2cm]{Stan_logo.png}
  \center
  \vspace{\baselineskip}
  \includegraphics[width=2.5cm]{stan_logo_wide.png}\\
  mc-stan.org
% \includegraphics[width=11cm]{mystery.jpeg}

\end{frame}

\begin{frame}
  \frametitle{Bayesian data analysis}  %
  \framesubtitle{Assessment}
  \begin{itemize}
  \item Exercises 2/3, and project work and presentation 1/3
     \begin{itemize}
     \item Minimum of 50\% of points must be obtained from both the project work and the exercises.
     % \item Preliminary grade boundaries\\
     %   <50\%=0, 50\%-60\%=1, 60\%-70\%=2, 70\%-80\%=3, 80\%-90\%=4, >90\%=5
     \end{itemize}
  \end{itemize}

\end{frame}

\begin{frame}
  \frametitle{Bayesian data analysis}  %

  \begin{itemize}
  \item Lectures describe basics and give broader overview (recorded
    and made available)
    \begin{itemize}
    \item written material has all the details and self-study
      is possible
    \end{itemize}
  \item Supporting material and assignments in
    {\small\url{https://avehtari.github.io/BDA_course_Aalto/Aalto2022.html}}
    \begin{itemize}
    \item reading instructions and chapter notes
    \item demos (very useful for assignments)
    \item slides (not very useful without the lectures)
    \item video clips
    \item links to additional material
    \end{itemize}
   \item R demos {\small\url{https://avehtari.github.io/BDA_course_Aalto/demos.html\#BDA_R_demos}}
  \item (Python demos {\small\url{https://avehtari.github.io/BDA_course_Aalto/demos.html\#BDA_Python_demos})}
  \item Aalto Zulip chat instance (link in MyCourses)
  \end{itemize}

\end{frame}

\begin{frame}
  \frametitle{Bayesian data analysis}  %
  \framesubtitle{Assignments}
  \begin{itemize}
  \item Weekly assignments (some have two weeks time)
    \begin{itemize}
    \item R (Python) simulation exercises
    \item Stan probabilistic programming exercises (via R (Python))
    \end{itemize}
  \item Related R (Python) demos available (see the course web site)
  \item TAs available: see Oodi for exercise sessions
  \item Exercise deadlines on Sunday (see detailed info in the course web page)
    \begin{itemize}
    \item we recommend to submit before Friday 3pm as TAs are not
      available during the weekend
    \item we allow the late submission on Sunday as some students are
      working on weekdays
    \end{itemize}
  \item After the exercise deadline, the grading period Monday--Tuesday
  \item Students grade 3 other exercises using peergrade.io
  \end{itemize}
  
\end{frame}

\begin{frame}
  \frametitle{Bayesian data analysis}  %
  \framesubtitle{R vs Python}

  \begin{itemize}
  \item We strongly recommend using R in the course as there are more
    packages for Stan and statistical analysis in general in R
  \item If you are already fluent in Python, but not in R, then using Python
    may be easier, but it can still be more useful to learn also R
  \end{itemize}
  
\end{frame}

\begin{frame}
  \frametitle{Bayesian data analysis}  %
  \framesubtitle{Assignments}
  \begin{itemize}
  \item Assignments are given on PeerGrade (also available in the course website)
  \item Assignments are returned and graded on Peergrade
  \end{itemize}
\end{frame}

\begin{frame}
  \frametitle{Assignments}  %
  \framesubtitle{peergrade.io}
  \begin{itemize}
  \item Peergrading used in BDA course since 2016
  \item Each student grades 3 exercises (randomly distributed)
  \item Detailed grading instructions -- rubric (available also on the course website)
  \item Also text feedback
  \item Possible to flag inappropriate grading (please, be polite!)
  \item TAs check flagged gradings
  \item Possible to give thumb up for great feedback
    \begin{itemize}
    \item those who give good feedback will get bonus points
    \end{itemize}
  \item See more at
    \url{https://avehtari.github.io/BDA_course_Aalto/assignments.html}
  \end{itemize}
  
\end{frame}

\begin{frame}
  \frametitle{Assignments}  %
  \framesubtitle{peergrade.io}

  \begin{itemize}
  \item Combined score: 70\% submission performance, 30\% feedback performance
    \pause
  \item Hand-in score:
    \begin{itemize}
    \item averaging the scores from peers
    \item after flagging, teacher may overrule the score
    \item different exercises have different weights
    \end{itemize}
    See details at \url{http://help.peergrade.io/interfaces-and-features/grading-and-scores/the-hand-in-score}
    \pause
  \item Feedback score:
    \begin{itemize}
    \item When students receive a review, they are asked to react to
      it using a scale ranging from ``Not useful at all'' to ``Extremely
      useful''.
    \item These ratings each correspond to a score between 0\% and 100\%.
    \item The feedback score is the average of the reaction scores.
    \item ``Somewhat useful. Could be more elaborate.'' is the
      baseline reaction.
    \end{itemize}
    
  \end{itemize}
  
\end{frame}

\begin{frame}
  \frametitle{Peergrade.io}  %
  \framesubtitle{Registration}
  \begin{itemize}
  \item Go to BDA MyCourses page
  \item Click Peergrade and login with Aalto account
    % \begin{itemize}
    % \item Aalto is getting campus license and integration to Oodi, but
    %   lawyers are still checking data protection issues
    % \end{itemize}
  \end{itemize}
  
\end{frame}

\begin{frame}
  \frametitle{Assignments}  %
  \framesubtitle{Plagiarism and empty reports}
  \begin{itemize}
  \item It's ok to discuss assignments with others
  \item It's ok to use code from the demos (good to mention the source)
  \item Don't copy reports from others or from internet
  \item Don't submit empty, almost empty or nonsense report
    \begin{itemize}
    \item these will be problematic for other students
    \item if you see such, you can mark it as problematic and get
      another one for grading
    \end{itemize}
  \end{itemize}
  
\end{frame}


\begin{frame}
  \frametitle{Project work}  %
  \framesubtitle{}
  \begin{itemize}
  \item Project work in groups of 1--3
    \begin{itemize}
    \item combines all the pieces learned in one project work
    \item R or Python notebook report
    \item project report peer graded
    \item oral presentation graded by me and TAs
    \end{itemize}
  \item More about projects later
  \end{itemize}
  
\end{frame}

\begin{frame}

  \frametitle{Zulip chat}  %
  \framesubtitle{bda2022.zulip.cs.aalto.fi}
  
\end{frame}

\begin{frame}

  \frametitle{RStudio, Quarto, R markdown}  %
  \framesubtitle{}

  \begin{itemize}
  \item RStudio is a great IDE for R
  \item Quarto is a new markdown language for making reports mixing
    text, code, equations, tables, etc
    \begin{itemize}
    \item Quarto is an improved variant from R markdown
    \end{itemize}
  \item RStudio has also visual editor for Quarto (and R markdown)
    making it easy for new users
  \item RStudio is also installed in Aalto JupyterHub
  \end{itemize}
  
\end{frame}  

\end{document}

%%% Local Variables:
%%% mode: latex
%%% TeX-master: t
%%% End:
