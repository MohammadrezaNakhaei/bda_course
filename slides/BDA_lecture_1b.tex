\documentclass[english,t]{beamer}
%\documentclass[finnish,english,handout]{beamer}

\usepackage[T1]{fontenc}
\usepackage[utf8]{inputenc}
\usepackage{times}
\usepackage{amsmath}
\usepackage[varqu,varl]{inconsolata} % typewriter
\usepackage{microtype}
\usepackage{url}
\urlstyle{same}

\mode<presentation>
{
  \setbeamercovered{invisible}
  \setbeamertemplate{itemize items}[circle]
  \setbeamercolor{frametitle}{bg=white,fg=navyblue}
  \setbeamertemplate{navigation symbols}{}
  \setbeamertemplate{headline}[default]{}
  \setbeamertemplate{footline}[split]
% Uncomment if want to show notes
%\setbeameroption{show notes}
}

\pdfinfo{            
  /Title      (BDA, Lecture 1, Practicalities) 
  /Author     (Aki Vehtari) % 
  /Keywords   (Bayesian data analysis)
}

\definecolor{navyblue}{rgb}{0,0,0.5}
% \definecolor{midnightblue}{rgb}{0.0977,0.0977,0.4375}
% \definecolor{lightgray}{rgb}{0.95,0.95,0.95}
% \definecolor{section}{rgb}{0,0.2549,0.6784}
% \definecolor{list1}{rgb}{0,0.2549,0.6784}
% \renewcommand{\emph}[1]{\textcolor{navyblue}{#1}}

\DeclareMathOperator{\E}{E}
\DeclareMathOperator{\Var}{Var}
\DeclareMathOperator{\var}{var}
\DeclareMathOperator{\Sd}{Sd}
\DeclareMathOperator{\sd}{sd}
\DeclareMathOperator{\Bin}{Bin}
\DeclareMathOperator{\Beta}{Beta}
\DeclareMathOperator{\logit}{logit}
\DeclareMathOperator{\N}{N}
\DeclareMathOperator{\U}{U}
\DeclareMathOperator{\BF}{BF}
%\DeclareMathOperator{\Pr}{Pr}
\def\euro{{\footnotesize \EUR\, }}
\DeclareMathOperator{\rep}{\mathrm{rep}}


\title[]{Bayesian data analysis}
\subtitle{Practical matters}

\author{Aki Vehtari}

\institute[Aalto University]{}

\begin{document}

\begin{frame}
  \frametitle{Bayesian data analysis (Aalto fall 2023)}  %
  \framesubtitle{}
  \begin{itemize}
  \item Book: Gelman, Carlin, Stern, Dunson, Vehtari \& Rubin: Bayesian Data
    Analysis, Third Edition. {\footnotesize (online PDF available)}
  \item The course website has more detailed information than these slides\\
    {\small\url{https://avehtari.github.io/BDA_course_Aalto/Aalto2023.html}}
  \item Timetable: see the course website
  \item TAs: David Kohns, Noa Kallioinen, Andrew Johnson, Leevi
    Lindgren, Anna Riha, Niko Siccha, Maksim Sinelnikov, Teemu
    Säilynoja
    \end{itemize}
    \vspace{-0.5\baselineskip}
 \begin{center}
   \includegraphics[width=2.6cm]{figs/BDA3.jpg}
 \end{center}

\end{frame}

\begin{frame}
  \frametitle{Bayesian data analysis}  %
  \framesubtitle{Pre-requisites}
  \begin{itemize}
  \item Basic terms of probability theory
    \begin{itemize}
    \item probability, probability density, distribution
    \item sum, product rule, and Bayes' rule
    \item expectation, mean, variance, median
    \end{itemize}
  \item Some algebra and calculus
  \item Basic visualization techniques (R or Python)
    \begin{itemize}
    \item histogram, density plot, scatter plot
    \end{itemize}
  \end{itemize}

  These will be tested with the first assignment round

\end{frame}

\begin{frame}
  \frametitle{Bayesian data analysis}  %
  \framesubtitle{Pre-requisites}
  \begin{itemize}
  \item What to do if the course seems to be too difficult
    \begin{itemize}
    \item refresh your memory on pre-requisites (see the course web
      site for some links)
    \item ask for help
    \item consider reading Regression and Other Stories \url{https://avehtari.github.io/ROS-Examples/}
    \item consider reading Statistical rethinking + watching videos \url{https://xcelab.net/rm/statistical-rethinking/}
    \end{itemize}
  \end{itemize}

\end{frame}

\begin{frame}
  \frametitle{Bayesian data analysis}  %
  \framesubtitle{Course contents}
  \begin{itemize}
  \item Background (Ch 1)
  \item Model, likelihood, prior, posterior, integration (Ch 2)
  \item Integration in multiparameter models (Ch 3)
  \item Basic integration methods (Ch 10)
  \item Markov chain Monte Carlo integration (Ch 11--12)
  \item Stan and probabilistic programming
  \item Hierarchical models (Ch 5)
  \item Model checking (Ch 6)
  \item Evaluating and comparing models (Ch 7 + extra material)
  \item Decision analysis (Ch 9)
  \item Large sample properties and Laplace approximation (Ch 4)
  \item Bayesian workflow (project)
  \end{itemize}
  
\end{frame}


\begin{frame}
  \frametitle{Bayesian data analysis}  %
  \framesubtitle{Different learning styles}

  \begin{itemize}
  \item Reading
  \item Listening lectures
  \item Solving problems
    \begin{itemize}
    \item mathematical derivations
    \item programming
    \end{itemize}
  \end{itemize}
  
\end{frame}

\begin{frame}
  \frametitle{Bayesian data analysis}  %
  \framesubtitle{Assessment}
  \begin{itemize}
  \item Assignments 60\%, and project work and presentation 40\%
     \begin{itemize}
     \item Minimum of 50\% of points must be obtained from both the project work and the assignments.
     \end{itemize}
  \end{itemize}

\end{frame}

\begin{frame}
  \frametitle{Bayesian data analysis}  %

  \begin{itemize}
  \item Lectures describe basics and give broader overview (recorded
    and made available)
    \begin{itemize}
    \item written material has all the details and self-study
      is possible
    \end{itemize}
  \item Supporting material and assignments in
    {\small\url{https://avehtari.github.io/BDA_course_Aalto/Aalto2023.html}}
    \begin{itemize}
    \item reading instructions and chapter notes
    \item demos (very useful for assignments)
    \item slides (not very useful without the lectures)
    \item video clips
    \item links to additional material
    \end{itemize}
   \item R demos {\small\url{https://avehtari.github.io/BDA_course_Aalto/demos.html\#BDA_R_demos}}
  \item (Python demos {\small\url{https://avehtari.github.io/BDA_course_Aalto/demos.html\#BDA_Python_demos})}
  \item Aalto Zulip chat instance (link in MyCourses)
  \end{itemize}

\end{frame}

\begin{frame}
  \frametitle{Bayesian data analysis}  %
  \framesubtitle{Assignments}
  \begin{itemize}
  \item Weekly assignments (some have two weeks time)
    \begin{itemize}
    \item R (Python) simulation assignments
    \item Stan probabilistic programming assignments (via R (Python))
    \end{itemize}
  \item Related R (Python) demos available (see the course web site)
  \item TAs available: the web page for TA session times
  \item Assignment deadlines on Sunday (see detailed info in the course web page)
    \begin{itemize}
    \item we recommend to submit before Friday 3pm as TAs are not
      available during the weekend
    \item we allow the late submission on Sunday as some students are
      working on weekdays
    \end{itemize}
  \item After the assignment deadline, the grading period Monday--Tuesday
  \item Students grade 3 other assignments using peergrade.io
  \end{itemize}
  
\end{frame}

\begin{frame}
  \frametitle{Bayesian data analysis}  %
  \framesubtitle{R vs Python}

  \begin{itemize}
  \item We strongly recommend using R in the course as there are more
    packages for Stan and statistical analysis in general in R
  \item If you are already fluent in Python, but not in R, then using Python
    may be easier, but it can still be more useful to learn also R
  \end{itemize}
  
\end{frame}

\begin{frame}
  \frametitle{Bayesian data analysis}  %
  \framesubtitle{Assignments}
  \begin{itemize}
  \item Assignments are available in the course website
  \item Assignments are returned and graded in Peergrade
  \end{itemize}
\end{frame}

\begin{frame}
  \frametitle{Assignments}  %
  \framesubtitle{peergrade.io}
  \begin{itemize}
  \item Peergrading used in BDA course since 2016
  \item Each student grades 3 assignments (randomly distributed)
  \item Detailed grading instructions -- rubric (available also on the course website)
  \item Also text feedback
  \item Possible to flag inappropriate grading (please, be kind!)
  \item TAs check flagged gradings
  \item Possible to give thumb up for great feedback
    \begin{itemize}
    \item those who give good feedback will get bonus points
    \end{itemize}
  \item See more at
    \url{https://avehtari.github.io/BDA_course_Aalto/assignments.html}
  \end{itemize}
  
\end{frame}

\begin{frame}
  \frametitle{Assignments}  %
  \framesubtitle{peergrade.io}

  \begin{itemize}
  \item Combined score: 80\% submission performance, 20\% feedback performance
    \pause
  \item Hand-in score:
    \begin{itemize}
    \item averaging the scores from peers
    \item after flagging, teacher may overrule the score
    \item different assignments have different weights
    \end{itemize}
    See details at \url{http://help.peergrade.io/interfaces-and-features/grading-and-scores/the-hand-in-score}
    \pause
  \item Feedback score:
    \begin{itemize}
    \item When students receive a review, they are asked to react to
      it using a scale ranging from ``Not useful at all'' to ``Extremely
      useful''.
    \item These ratings each correspond to a score between 0\% and 100\%.
    \item The feedback score is the average of the reaction scores.
    \item ``Somewhat useful. Could be more elaborate.'' is the
      baseline reaction.
    \end{itemize}
    
  \end{itemize}
  
\end{frame}

\begin{frame}
  \frametitle{Peergrade.io}  %
  \framesubtitle{Registration}
  \begin{itemize}
  \item Go to BDA MyCourses page
  \item Click Peergrade and login with Aalto account
  \end{itemize}
  
\end{frame}

\begin{frame}
  \frametitle{Assignments}  %
  \framesubtitle{Plagiarism and empty reports}
  \begin{itemize}
  \item It's OK to discuss assignments with others
  \item It's OK to use code from the demos (mention the source)
  \item It's OK to use AI, but need to mention when and how used
    \begin{itemize}
    \item Warning: I have tested these and they can provide very vague
      or completely wrong results for the course contents
    \item Might be most useful for getting ideas for code and markdown syntax
    \end{itemize}
  \item Don't copy reports from others or from internet
  \item Don't submit empty, almost empty or nonsense report
    \begin{itemize}
    \item these will be problematic for other students
    \item if you see such, send TAs a message and mark it as
      problematic in Peergrade and get another one for grading
    \end{itemize}
  \end{itemize}
  
\end{frame}


\begin{frame}
  \frametitle{Project work}  %
  \framesubtitle{}
  \begin{itemize}
  \item Project work in groups of 1--3
    \begin{itemize}
    \item combines all the pieces learned in one project work
    \item R or Python notebook report
    \item project report peer graded (40\% of the project score)
    \item oral presentation graded by me and TAs (60\% of the project score)
    \end{itemize}
  \item More about projects later
  \end{itemize}
  
\end{frame}

\begin{frame}

  \frametitle{Zulip chat}  %
  \framesubtitle{bda2023.zulip.cs.aalto.fi}

  \begin{itemize}
  \item Aalto login, hosted by Aalto IT, deleted after one year
  \item The web interface is better, but the mobile app has gained
    push notifications, too
  \item Different streams for announcements, general, assignments, etc.
  \end{itemize}
  
\end{frame}

\begin{frame}

  \frametitle{RStudio, Quarto, R markdown}  %
  \framesubtitle{}

  \begin{itemize}
  \item RStudio is a great IDE for R
  \item Quarto is a new markdown language for making reports mixing
    text, code, equations, tables, etc
    \begin{itemize}
    \item \textit{Quarto is the next iteration of R Markdown, and
        allows you can create dynamic content with Python, R, Julia,
        and Observable, author documents as plain text markdown or
        Jupyter notebooks, and output to multiple format types.}
    \end{itemize}
  \item RStudio has also visual editor for Quarto (and R markdown)
    making it easy for new users
  \item RStudio is also installed in Aalto JupyterHub
  \end{itemize}
  
\end{frame}  

\begin{frame}

  \frametitle{jupyter.cs.aalto.fi}  %
  \framesubtitle{}

  \begin{itemize}
  \item No need to install anything locally, everything can be done in
    Aalto JupyterHub
  \item There is some support for local installations (see FAQ in the
    course web page)
  \end{itemize}
  
\end{frame}  

\begin{frame}

  \frametitle{FAQ}  %
  \framesubtitle{}

  \begin{itemize}
  \item {\small\url{https://avehtari.github.io/BDA_course_Aalto/FAQ.html}}
  \item For example,
    \begin{itemize}
    \item R packages used in demos
    \item Installing aaltobda package
    \item Installation problems
    \item Remote access
    \item Tidyverse and pipes
    \item I missed some deadline or wasn’t able to do some part of the course
    \end{itemize}
  \end{itemize}
  
\end{frame}  

\end{document}

%%% Local Variables:
%%% mode: latex
%%% TeX-master: t
%%% End:
